\documentclass[10pt]{beamer}

\usepackage[utf8]{inputenc}
\usepackage[spanish]{babel}

\usepackage{graphicx}
\usepackage{multimedia}
\graphicspath{{./multimedia/}}

\usepackage{hyperref}

\immediate\write18{./Multimedia/tor\_status.sh > tor\_status.tex}
\immediate\write18{./Multimedia/tor\_flow.sh > tor\_flow.tex}

\usetheme[progressbar=frametitle]{metropolis}
\usepackage{appendixnumberbeamer}

\usepackage{booktabs}
\usepackage[scale=2]{ccicons}

\usepackage{pgfplots}
\usepgfplotslibrary{dateplot}

\usepackage{xspace}

\title{Tor, ¿la herramienta definitiva de anonimato?}
\subtitle{Anonimato en la red}
\date{}
\author{Ignacio Aguilera Martos}


\begin{document}
	

	
	
%Diapositiva de Título.
\frame{\titlepage}

%%%%%%%%%%%%%%%%%%%%%%%%%%%%%%%%%%%%%%%%%%%%%%%%%%%%%%%%%%%%%%%%%%%%%%%%%%%%%%%%%%%%%%%%%
%%%%%%%%%%%%%%%%%%%%%%%%%%%%%%%%%%%%Parte de Tor%%%%%%%%%%%%%%%%%%%%%%%%%%%%%%%%%%%%%%%%%
%%%%%%%%%%%%%%%%%%%%%%%%%%%%%%%%%%%%%%%%%%%%%%%%%%%%%%%%%%%%%%%%%%%%%%%%%%%%%%%%%%%%%%%%%
	
%Índice
\begin{frame}{Contenidos}
	\setbeamertemplate{section in toc}[sections numbered]
	\tableofcontents[hideallsubsections]
\end{frame}

%Sección de introducción al anonimato
\section{¿Por qué es importante el anonimato?}

\begin{frame}{Anonimato}
	\begin{itemize}
		\item Debemos ser dueños de nuestra información. \pause
		\item No somos manipulados en función de nuestros datos si permanecemos anónimos. \pause
	\end{itemize}
	\metroset{block=fill}
	\begin{block}{Privacidad en Internet}
		\pause La privacidad en Internet se refiere al derecho de la privacidad personal en relación con el almacenamiento, la reutilización, la provisión a terceros y la exhibición de información relativa a uno mismo a través de Internet.
	\end{block}
	\pause
	\begin{alertblock}{Charla de Introducción a la privacidad}
		$https://bitbucket.org/josealberto4444/charla_introduccion_privacidad$
	\end{alertblock}
\end{frame}




	
\end{document}